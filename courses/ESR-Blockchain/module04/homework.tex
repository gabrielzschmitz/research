% This template is based on Pascal Michaillat Minimalist LaTeX Template for
% Academic Papers: https://github.com/pmichaillat/latex-paper
\documentclass[letterpaper,11pt,leqno]{article}
\usepackage{homework}
\usepackage{float}
\usepackage{cite}
\usepackage{siunitx}
\usepackage{array}
\usepackage{booktabs}
\bibliographystyle{bibliography}

% Paper title to populate PDF metadata:
\hypersetup{pdftitle={Resumo: Blockchain GoLedger Módulo 4}}

% Path to BibTeX file with references:
\newcommand{\bib}{bibliography}

% Path to PDF file with figures:
\newcommand{\pdf}{figures/figures}

% To use double quotes
\newcommand{\quotes}[1]{``#1''}

% Defining problem, subproblem and solution:
\newtheorem{problem}{\refstepcounter{problemcount}{Problema}}
\newtheorem{subproblem}{\refstepcounter{subproblemcount}}[problem]
\makeatletter
\renewcommand{\thesubproblem}{%
  \theproblem-\ifnum\value{subproblem}<27%
    \alph{subproblem}%
  \else%
    \number\numexpr\value{subproblem}-26\relax%
  \fi%
}
\makeatother
\newenvironment{solution}[1][\it{Solução}]{
  \refstepcounter{solutioncount}{\textbf{#1:}\\}}
{}

% Adding code support
\usepackage{listings}
\usepackage{xcolor}
\definecolor{darkBackground}{RGB}{40, 44, 52} % Background color
\definecolor{lightGray}{RGB}{55, 61, 69}      % Light gray for the frame
\definecolor{blue}{RGB}{66, 134, 244}         % Blue for keywords
\definecolor{green}{RGB}{80, 250, 123}        % Green for comments
\definecolor{red}{RGB}{255, 85, 85}           % Red for strings
\definecolor{purple}{RGB}{189, 147, 249}      % Purple for other keywords
\definecolor{orange}{RGB}{255, 184, 108}      % Orange for constants
\lstdefinestyle{oneDarkStyle}{
    backgroundcolor=\color{darkBackground}, % Solid background
    basicstyle=\ttfamily\footnotesize\color{white}, % Font style
    breaklines=false,
    frame=single,
    captionpos=b,
    rulecolor=\color{darkBackground}, % Change frame color to match background
    keywordstyle=\color{blue}\bfseries,
    commentstyle=\color{green},
    stringstyle=\color{red},
    numberstyle=\tiny\color{black},
    showtabs=false,                  
    numbers=left,
    morekeywords={int, float, return, void}, % Extra keywords
}
\lstset{style=oneDarkStyle}

\begin{document}

% Title:
\title{Tarefa: Blockchains Corporativos\\ Hyperledger Fabric\\ Módulo 4}

% Authors:
\author{Gabriel dos Santos Schmitz}

% Date:
\date{Dezembro 2024}

\begin{titlepage}

	\maketitle

\end{titlepage}

% Tarefa 1
\section{Analisando Rede Fabric}

% 1
\begin{problem}
Considere que um chaincode instalado na rede tenha a seguinte regra de consenso:

\begin{lstlisting}[language=SQL, caption=Chaincode]
OUTOF( 2,
  AND(org1.member, OR(org2.member, org4.member)),
  OUTOF(1, org3.member, AND(org4.member, org5.member)),
  OR(org3.member, OUTOF(1, org1.member, org2.member)
)
\end{lstlisting}

A tabela abaixo apresenta transações feitas no sistema e quais organizações a
aprovaram:

\begin{table}[h!]
	\centering
	\begin{tabular}{cccccc}
		\toprule
		\textbf{Transação} & \textbf{Org1} & \textbf{Org2} & \textbf{Org3} & \textbf{Org4} & \textbf{Org5} \\
		\midrule
		1                  & Sim           & Sim           & Sim           & Não           & Não           \\
		2                  & Não           & Não           & Não           & Sim           & Sim           \\
		3                  & Não           & Não           & Sim           & Sim           & Sim           \\
		4                  & Sim           & Não           & Não           & Não           & Não           \\
		5                  & Não           & Não           & Sim           & Não           & Não           \\
		6                  & Sim           & Sim           & Não           & Sim           & Não           \\
		7                  & Sim           & Não           & Não           & Não           & Sim           \\
		8                  & Sim           & Não           & Não           & Sim           & Não           \\
		9                  & Não           & Sim           & Não           & Sim           & Não           \\
		10                 & Não           & Sim           & Não           & Não           & Sim           \\
		\bottomrule
	\end{tabular}
	\caption{Tabela de Transações e Organizações}
	\label{tab:transacoes}
\end{table}

\end{problem}

% 1-a
\begin{subproblem}
	Quais transações descritas na tabela acima foram aprovadas?
\end{subproblem}
\begin{solution}
	Transações 1, 3, 5, 6, 9 foram aprovadas.
\end{solution}

% 1-b
\begin{subproblem}
	Qual a entidade da rede Fabric responsável por conferir se o consenso do
	chaincode foi atingido?
\end{subproblem}
\begin{solution}
	A entidade responsável por conferir se o consenso do chaincode foi atingido no
	Hyperledger Fabric é o peer validatório (validating peer).
\end{solution}

\newpage
% 1-c
\begin{subproblem}
	Se a transação 10 fosse uma transação envolvendo uma coleção privada entre a
	Org2 e Org5, o resultado da transação sofreria alguma mudança? Se sim, porquê?
\end{subproblem}
\begin{solution}
	Sim, o resultado da transação 10 sofreria uma mudança. Se fosse uma transação
	envolvendo uma coleção privada entre as organizações Org2 e Org5, ela seria
	aprovada, mesmo que não atendesse ao consenso do chaincode geral, porque as
	regras de aprovação para coleções privadas são definidas de forma independente
	do chaincode principal.
\end{solution}

% \bibliography{bibliography}

\end{document}
