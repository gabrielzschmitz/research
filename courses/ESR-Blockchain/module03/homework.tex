% This template is based on Pascal Michaillat Minimalist LaTeX Template for
% Academic Papers: https://github.com/pmichaillat/latex-paper
\documentclass[letterpaper,11pt,leqno]{article}
\usepackage{homework}
\usepackage{float}
\usepackage{cite}
\usepackage{siunitx}
\usepackage{array}
\usepackage{booktabs}
\bibliographystyle{bibliography}

% Paper title to populate PDF metadata:
\hypersetup{pdftitle={Resumo: Blockchain GoLedger Módulo 3}}

% Path to BibTeX file with references:
\newcommand{\bib}{bibliography}

% Path to PDF file with figures:
\newcommand{\pdf}{figures/figures}

% To use double quotes
\newcommand{\quotes}[1]{``#1''}

% Defining problem, subproblem and solution:
\newtheorem{problem}{\refstepcounter{problemcount}{Problema}}
\newtheorem{subproblem}{\refstepcounter{subproblemcount}}[problem]
\makeatletter
\renewcommand{\thesubproblem}{%
  \theproblem-\ifnum\value{subproblem}<27%
    \alph{subproblem}%
  \else%
    \number\numexpr\value{subproblem}-26\relax%
  \fi%
}
\makeatother
\newenvironment{solution}[1][\it{Solução}]{
  \refstepcounter{solutioncount}{\textbf{#1:}\\}}
{}

\begin{document}

% Title:
\title{Tarefa: Blockchains Corporativos\\ Hyperledger Fabric\\ Módulo 3}

% Authors:
\author{Gabriel dos Santos Schmitz}

% Date:
\date{Novembro 2024}

\begin{titlepage}

	\maketitle

\end{titlepage}

% Tarefa 1
\section{Analisando Rede Fabric}

% 1
\begin{problem}
Para essa tarefa, considera a representação gráfica simplificada de uma rede
Fabric mostrada na imagem abaixo.
\end{problem}

\begin{figure}[H]
	{\includegraphics[scale=0.5,page=2]{\pdf}}
	%\caption{}
	%\note{}
	\label{f:figure1}\end{figure}

% 1-a
\begin{subproblem}
	Qual dos canais criados permite uma comunicação entre todas as organizações do
	sistema?
\end{subproblem}
\begin{solution}
	O canal que permite comunicação entre todas as organições é o canal
	\textbf{C1}.
\end{solution}

% 1-b
\begin{subproblem}
	Quais canais a organização 4 têm acesso?
\end{subproblem}
\begin{solution}
	A organização 4 tem acesso aos canais \textbf{C1} e \textbf{C2}.
\end{solution}

% 1-c
\begin{subproblem}
	Há um interesse dos administradores desta rede de trocar informações restritas
	entre as organizações 1 e 3? O que pode ser feito para atingir esse objetivo?
\end{subproblem}
\begin{solution}
	Não há interesse em trocar informações restritivas entre orgs 1 e 3 nesta
	configuração de rede. Para isso, seria necessário criar outro canal
	(\textbf{C4}) onde apenas os peers 1 e 3 tenham acesso.
\end{solution}

% \bibliography{bibliography}

\end{document}
