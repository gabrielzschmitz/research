% This template is based on Pascal Michaillat Minimalist LaTeX Template for
% Academic Papers: https://github.com/pmichaillat/latex-paper
\documentclass[letterpaper,11pt,leqno]{article}
\usepackage{homework}
\usepackage{float}
\usepackage{cite}
\usepackage{siunitx}
\bibliographystyle{bibliography}

% Paper title to populate PDF metadata:
\hypersetup{pdftitle={Resumo: Blockchain GoLedger Módulo 1}}

% Path to BibTeX file with references:
\newcommand{\bib}{bibliography}

% Path to PDF file with figures:
\newcommand{\pdf}{figures/figures}

% To use double quotes
\newcommand{\quotes}[1]{``#1''}

% Defining problem, subproblem and solution:
\newtheorem{problem}{\refstepcounter{problemcount}{Problema}}
\newtheorem{subproblem}{\refstepcounter{subproblemcount}}[problem]
\makeatletter
\renewcommand{\thesubproblem}{%
  \theproblem-\ifnum\value{subproblem}<27%
    \alph{subproblem}%
  \else%
    \number\numexpr\value{subproblem}-26\relax%
  \fi%
}
\makeatother
\newenvironment{solution}[1][\it{Solução}]{
  \refstepcounter{solutioncount}{\textbf{#1:}\\}}
{}

\begin{document}

% Title:
\title{Tarefa: Blockchains Corporativos\\ Hyperledger Fabric\\ Módulo 1}

% Authors:
\author{Gabriel dos Santos Schmitz}

% Date:
\date{Outubro 2024}

\begin{titlepage}

	\maketitle

\end{titlepage}

% Tarefa 1
\section{Tokens UTXO}

% 1
\begin{problem}
Considere que o usuário \textit{A} tenha três tokens no formato UTXO com
valores de 127 GLD, 5 GLD e 62 GLD.\@ Esse usuário então deseja fazer uma
transferência de 140 GLD para o usuário \textit{B}.
\end{problem}

% 1-a
\begin{subproblem}
	Quais tokens do usuário \textit{A} deverão ser utilizados como input na
	transação, utilizando o menor número de tokens possíveis para isso? O que
	ocorrerá com esses tokens utilizados como input no final da transação?
\end{subproblem}
\begin{solution}
	Para realizar a transação com o menor número de tokens possível, o usuário
	\textit{A} deverá utilizar o token de 127 GLD inteiro e 13 GLD do token de 62
	GLD, restando 49 GLD nesse último token.\@ Ao final da transação, o primeiro
	token será transferido de \textit{A} para \textit{B} junto com os 13 GLD do
	segundo token.
\end{solution}

% 1-b
\begin{subproblem}
	Considerando sua resposta da questão 1, qual será o output desta transação?
\end{subproblem}
\begin{solution}
	O output da transação será um token de 127 GLD e um token de 13 GLD
	transferidos de \textit{A} para \textit{B}. Além disso, \textit{A} manterá o
	saldo remanescente de 49 GLD no segundo token e os 5 GLD do terceiro token.
\end{solution}

% Tarefa 2
\section{Blockchain Demo}

% 2
\begin{problem}
Utilize o site https://andersbrownworth.com/blockchain/blockchain para responder
as questões a seguir. Se familiarize com o site antes de respondê-las.
\end{problem}

% 2-a
\begin{subproblem}
	O valor de nonce, gerado após utilizar do botão de minerar, garante que o
	valor de hash do bloco seja válido. No exemplo do site, qual característica é
	esperada para o valor da hash para que a rede de blocos seja válida?
\end{subproblem}
\begin{solution}
	O hash deve começar com quatro dígitos zero para que o bloco seja considerado
	válido.
\end{solution}

% 2-b
\begin{subproblem}
	Para o bloco com o valor Prévio igual a 0, considere o número do bloco como
	2461 e seus dados como \underline{5 BTC -> Maria}. Qual o valor da hash desse
	bloco, minerado com o menor valor inteiro positivo possível para o nonce.
\end{subproblem}
\begin{solution}
	O hash resultante é: \\
	0000417825649abc576be53eef5b026c80643738d500ef025aef48612be8c09d, obtido com o
	menor valor de nonce igual a 146285.
\end{solution}

% \bibliography{bibliography}

\end{document}
