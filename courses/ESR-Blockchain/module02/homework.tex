% This template is based on Pascal Michaillat Minimalist LaTeX Template for
% Academic Papers: https://github.com/pmichaillat/latex-paper
\documentclass[letterpaper,11pt,leqno]{article}
\usepackage{homework}
\usepackage{float}
\usepackage{cite}
\usepackage{siunitx}
\usepackage{array}
\usepackage{booktabs}
\bibliographystyle{bibliography}

% Paper title to populate PDF metadata:
\hypersetup{pdftitle={Resumo: Blockchain GoLedger Módulo 2}}

% Path to BibTeX file with references:
\newcommand{\bib}{bibliography}

% Path to PDF file with figures:
\newcommand{\pdf}{figures/figures}

% To use double quotes
\newcommand{\quotes}[1]{``#1''}

% Defining problem, subproblem and solution:
\newtheorem{problem}{\refstepcounter{problemcount}{Problema}}
\newtheorem{subproblem}{\refstepcounter{subproblemcount}}[problem]
\makeatletter
\renewcommand{\thesubproblem}{%
  \theproblem-\ifnum\value{subproblem}<27%
    \alph{subproblem}%
  \else%
    \number\numexpr\value{subproblem}-26\relax%
  \fi%
}
\makeatother
\newenvironment{solution}[1][\it{Solução}]{
  \refstepcounter{solutioncount}{\textbf{#1:}\\}}
{}

\begin{document}

% Title:
\title{Tarefa: Blockchains Corporativos\\ Hyperledger Fabric\\ Módulo 2}

% Authors:
\author{Gabriel dos Santos Schmitz}

% Date:
\date{Novembro 2024}

\begin{titlepage}

	\maketitle

\end{titlepage}

% Tarefa 1
\section{Chaves Públicas e Privadas}

% 1
\begin{problem}
O site https://andersbrownworth.com/blockchain/public-private-keys/transaction
contém ferramentas interativas para demonstrar o funcionamento de chaves
públicas e privadas, e também a utilização delas para assinar transações.
Explore o site e o utilize para responder às questões abaixo.
\end{problem}

% 1-a
\begin{subproblem}
	Qual o valor da assinatura de uma transação com os dados da tabela abaixo?
	\begin{table}[ht]
		\centering
		\resizebox{\textwidth}{!}{ % Resize the table to fit the text width
			\begin{tabular}{@{}lp{11cm}@{}} % Set a width for the second column
				\toprule
				\textbf{Valor}                                            & \textbf{De}                                                 \\ \midrule
				$35.00                                                    & 041e728b6c57cf4c6be2f2d593cfddad442523f7f316e483d0b2216b38f \\
				                                                          & dfd55652fb13d2b61983e696abf8e9badb1002e08e9bad1a882dd787fd  \\
				                                                          & c2598969c67bd                                               \\\bottomrule
				\textbf{Para}                                             & \textbf{Chave Privada}                                      \\ \midrule
				04cc955bf8e359cc7ebbb66f4c2dc616a93e8ba08e93d27996e20299b & 447045612320949259997020406888024561147803363458274417137   \\
				a92cba9cbd73c2ff46ed27a3727ba09486ba32b5ac35dd20c0adec020 & 05004851108031018522                                        \\
				536996ca4d9f3d74                                          &                                                             \\ \bottomrule
			\end{tabular}}
	\end{table}
\end{subproblem}
\begin{solution}
	O valor da assinatura baseado na transação é de: \\
	3045022100960016f1a50a1d36934f233a24b571ecadc7c045fd830c93b510b3f0f73f36 \\
	7a02205b56171bc8a545ec30d64b60f139fba98093f13deef7b9205e976d615fad8405
\end{solution}

% 1-b
\begin{subproblem}
	Usando os mesmos dados de De, Para e Chave Privada da tabela na questão 1,
	identifique quais das transações na tabela abaixo são válidas:
	\begin{table}[ht]
		\centering
		\resizebox{\textwidth}{!}{ % Resize the table to fit the text width
			\begin{tabular}{@{}lp{1cm}@{}} % Set a width for the second column
				\toprule
				\textbf{Tx} & \textbf{Valor} & \textbf{Assinatura}                                                                                                                              \\ \midrule
				1           & $5.00          & 304502210092d7a8c41c090fb9d26f495dc381297e17a3961cf2c60e7826179023bbaf6ef602201658154e537e55de6064f3a781902cef1541b7b75a29bbf3616d8d3aaba7eca8   \\
					\midrule
				2           & $3.00          & 3046022100befb38ad806de564f12f50db48027fe7ccdf5fcfabbb9f228055cd2bf8a80c2f022100b7f4e0283b1ba099a206e1328bb23ff7d2b413a9cbe8070ccb0dd83334bce53e \\
				\midrule
				3           & $9.00          & 304402200ff25c5e99b891f7f4ddf8c5acc18c1b0db52bccc9664bcee5b80452fb6ee99602203e499de01b0be560dc6d8efb9e0d03df6f011eb75d16359abf5bcc90bdfd09ff     \\
					\midrule
				4           & $1.00          & 3045022100d799c67ccb033646b284b3cdbbf133b437757c03564279bc6c7200b877910620022005a12c62b203da939b4201368b49fcdfaae95bc0bbac14f4b14eb77c30b84eb0   \\
				\midrule
				5           & $2.00          & 304502206fd040f1a2377bbb4e95a46ff60e9ea9a1de162739b527d287e551ad826cc89a022100ea7fc1ff3e4e05eb6211d9d07e94cd3c1d015bb8a9f48b1e40bd80d19518a0e2   \\
				\bottomrule
			\end{tabular}
		}
		\caption{Tabela de Transações}
		\label{tab:transacoes}
	\end{table}
\end{subproblem}
\begin{solution}
	Verificando as tranções com as assinaturas providas as transções validas são
	as de\\
	\text{Tx} = [1, 4, 5].
\end{solution}


% \bibliography{bibliography}

\end{document}
