% This template is based on Pascal Michaillat Minimalist LaTeX Template for
% Academic Papers: https://github.com/pmichaillat/latex-paper
\documentclass[letterpaper,11pt,leqno]{article}
\usepackage{summary}
\usepackage{float}
\usepackage{cite}
\usepackage{siunitx}
\bibliographystyle{bibliography}

% Paper title to populate PDF metadata:
\hypersetup{pdftitle={Resumo: Blockchain GoLedger Módulo 1}}

% Path to BibTeX file with references:
\newcommand{\bib}{bibliography}

% Path to PDF file with figures:
\newcommand{\pdf}{figures/figures}

% To use double quotes
\newcommand{\quotes}[1]{``#1''}

% Adding code support
\usepackage{listings}
\usepackage{xcolor}
\definecolor{darkBackground}{RGB}{40, 44, 52} % Background color
\definecolor{lightGray}{RGB}{55, 61, 69}      % Light gray for the frame
\definecolor{blue}{RGB}{66, 134, 244}         % Blue for keywords
\definecolor{green}{RGB}{80, 250, 123}        % Green for comments
\definecolor{red}{RGB}{255, 85, 85}           % Red for strings
\definecolor{purple}{RGB}{189, 147, 249}      % Purple for other keywords
\definecolor{orange}{RGB}{255, 184, 108}      % Orange for constants
\lstdefinestyle{oneDarkStyle}{
    backgroundcolor=\color{darkBackground}, % Solid background
    basicstyle=\ttfamily\footnotesize\color{white}, % Font style
    breaklines=false,
    frame=single,
    captionpos=b,
    rulecolor=\color{darkBackground}, % Change frame color to match background
    keywordstyle=\color{blue}\bfseries,
    commentstyle=\color{green},
    stringstyle=\color{red},
    numberstyle=\tiny\color{black},
    showtabs=false,                  
    numbers=left,
    morekeywords={int, float, return, void}, % Extra keywords
}
\lstset{style=oneDarkStyle}

\begin{document}

\begin{figure}[t]
	{\includegraphics[scale=0.25,page=1]{\pdf}}
	\vspace{-50pt}
	\label{f:logo}\end{figure}

% Title:
\title{Resumo: Blockchain GoLedger Módulo 3}

% Authors:
\author{Gabriel dos Santos Schmitz
	%
	% Affiliations and acknowledgements:
	\thanks{Gabriel dos Santos Schmitz: UTFPR. Eu agradeço colegas e orientador
		por comentários e discussões úteis. Este trabalho foi apoiado por Grupo PQC
		UTFPR}}

% Date:
\date{Novembro 2024}

% Permanent URL:
\available{https://github.com/gabrielzschmitz/research/tree/main/summaries/ESR-BlockchainCourse}

\begin{titlepage}

	\maketitle

	% Abstract:
	This document provides an in-depth overview of Hyperledger Fabric, a
	permissioned blockchain framework tailored for enterprise applications. It
	explores key components such as the Orderer, responsible for transaction
	ordering and network consensus; the Certification Authority (CA), which
	manages permissions through digital certificates; and the Membership Service
	Provider (MSP), ensuring secure authentication. The analysis covers the
	functionality of Chaincodes as smart contracts for asset management and the
	role of Clients in facilitating external interactions with the blockchain.
	Emphasis is placed on privacy features like Private Data Collections, enabling
	selective data sharing among organizations while maintaining transparency
	through cryptographic validation. The document concludes by highlighting the
	framework’s flexibility in governance and detailed permission control, making
	it a robust solution for complex business networks.

\end{titlepage}

% Topic 1
\section{Conceitos de Hyperledger Fabric}

\begin{figure}[H]
	{\includegraphics[scale=0.4,page=2]{\pdf}}
	\caption{Tecnologia integrada}
	%\note{}
	\label{f:figure1}\end{figure}

\paragraph{}
Em um blockchain não permissionado, qualquer pessoa pode criar nós na rede,
identificando-se por meio de pares de chaves criptográficas geradas
aleatoriamente. A rede utiliza ativos escassos, como criptomoedas, para realizar
transações, possibilitando o registro de eventos entre partes desconhecidas. O
uso da rede envolve custos pagos em criptomoedas, e o acesso ao ledger é feito
por meio de wallets.

\paragraph{}
Redes de blockchain permissionado são compostas por consórcios ou reguladores,
com as organizações participantes sendo identificadas criptograficamente. A
entrada na rede exige autorização, exceto para os fundadores. Essas redes
permitem que processos compartilhados sejam codificados em contratos
inteligentes e seguidos por regras de consenso, baseadas em leis e normas.
Existem diversos frameworks disponíveis para o desenvolvimento dessas redes.

\begin{figure}[H]
	{\includegraphics[scale=0.5,page=3]{\pdf}}
	\caption{Hyperledger Foundation}
	%\note{}
	\label{f:figure2}\end{figure}

\paragraph{}
A Hyperledger Foundation é um projeto de código aberto administrado pela Linux
Foundation, focado em aprimorar tecnologias blockchain para diferentes setores
empresariais. É o projeto de crescimento mais rápido na história da fundação,
contando com a colaboração de empresas financeiras, bancos, IoT, logística, e
indústrias tecnológicas.

\paragraph{}
O Hyperledger Project foi criado em 2015 com o objetivo de fomentar o uso de
blockchain em ambientes empresariais e industriais. O primeiro projeto foi o
Hyperledger Fabric, originalmente conhecido como OpenLedger e doado pela IBM. Os
projetos da Hyperledger são categorizados em Graduados, Incubados, e
Laboratório, podendo incluir frameworks, bibliotecas, ferramentas, ou clientes
de acesso.

\begin{figure}[H]
	{\includegraphics[scale=0.5,page=4]{\pdf}}
	\caption{Exemplo de Rede Hyperledger Fabric}
	%\note{}
	\label{f:figure3}\end{figure}

\paragraph{}
A rede Fabric é ideal para situações que envolvem processos com múltiplas
entidades ou organizações conhecidas e confiáveis, onde o registro de
informações segue um sistema de credenciais hierárquicas. A governança da rede
pode ser flexível, variando entre modelos democráticos e regulados, conforme a
necessidade das instituições envolvidas.

\section{Confiabilidade Hyperledger Fabric}

\paragraph{}
O Hyperledger Fabric utiliza o algoritmo Raft para alcançar consenso, que
envolve o compartilhamento de estado entre nós com tolerância a falhas. Este
modelo é semelhante ao consenso Proof of Authority, mas não é Bizantino (BFT).
Os nós de ordenação são responsáveis por manter o consenso, que pode ser
definido através de um chaincode ou canal. A infraestrutura também utiliza o
etcd, um armazenamento chave-valor distribuído.

\paragraph{}
No Hyperledger Fabric, a rede é permissionada, composta por consórcios ou redes
de negócios, onde cada organização é representada por uma Autoridade
Certificadora (CA). A confiabilidade é garantida pela comparação de informações
assinadas por diferentes organizações, utilizando contratos inteligentes
(chaincodes), que não são armazenados no ledger. Esse modelo se assemelha a
sistemas contratuais do mundo físico, como cartórios.

\newpage{}

\paragraph{}
O Ethereum pode ser utilizado tanto em redes permissionadas quanto não
permissionadas. Smart contracts são armazenados no ledger como bytecode, e o
consenso é alcançado via PoW, PoS ou PoA, com a EVM validando o hash do bytecode
antes da execução. Transações envolvem uma contagem de gas, e a reputação dos
nós é utilizada nos modelos PoS e PoA, garantindo confiabilidade. Apenas o
modelo PoA Clique não é BFT.

\section{Componentes Hyperledger Fabric}

\paragraph{}
No Hyperledger Fabric, um Asset é um elemento compartilhado na rede, podendo ser
tangível ou intangível. A rede é composta por Nodes, que são os dispositivos
participantes, e utiliza um Ledger, o livro-razão das transações. A
identificação dos membros é gerida pelo Membership Service Provider (MSP),
enquanto a CA (Autoridade Certificadora) garante a autenticidade das
identidades. A comunicação entre partes da rede ocorre através de Channels, que
possuem um ledger dedicado, e os Chaincodes atuam como contratos inteligentes. A
rede pode ser acessada por clientes externos, conhecidos como Clients, formando
uma Business Network.

\paragraph{}
O ledger no Hyperledger Fabric é imutável e privado, armazenando transações de
configuração e de aplicação de forma cronológica. As transações podem criar,
atualizar ou deletar assets, e são mantidas nos nós (peers e orderers). Um
Channel é um meio de comunicação entre membros específicos da rede, definido por
um conjunto de organizações, peers, um ledger, chaincodes, e nós de ordenação
que asseguram as regras do canal.

\paragraph{}
O controle de acesso é gerido através de uma Access Control List (ACL), permitindo o gerenciamento detalhado de leitura, escrita, e administração dos recursos da rede. As permissões são atribuídas por meio de Policies, como "MyPolicy", que pode permitir acesso com base nas assinaturas das organizações. As políticas podem ser aplicadas a channels, organizações, e aplicações, utilizando ferramentas como configtxgen e arquivos de configuração configtx.yaml.

\paragraph{}
Os assets na rede podem representar elementos tangíveis ou intangíveis,
armazenados em formatos binários ou JSON. Eles funcionam como entradas em uma
tabela de banco de dados, com estados mantidos na rede e modificados dentro de
channels. Cada asset é indexado por uma chave primária ou composta, facilitando
operações de criação, modificação, e consulta no ledger.

\newpage{}

\begin{lstlisting}[language=Go, caption=Asset Example]
// Description of a book
var Book = assets.AssetType{
  Tag: "book",
  Label: "Book",
  Description: "Book",
  Props: []assets.AssetProp{
    {
      IsKey: true, // Primary Key
      Tag: "title",
      Label: "Book Title",
      DataType: "string",
      Writers: []string{`org2MSP`}, // This means only org2 can
                                    // create the asset (others can edit)
    },
    {
      Tag: "currentTenant",
      Label: "Current Tenant",
      DataType: "->person", /// Reference to another asset
    },
    {
      Tag: "genres",
      Label: "Genres",
      DataType: "[]string", // String list
    },
    {
      Tag: "published",
      Label: "Publishment Date",
      DataType: "datetime", // Date property
    },
  },
}
\end{lstlisting}

\paragraph{}
Nodes são dispositivos que armazenam ledgers e contratos inteligentes na rede
Hyperledger Fabric. Existem dois tipos principais de nodes: Peers, que pertencem
a uma organização e mantêm cópias do ledger, e Ordering Nodes, que recebem
transações, realizam o consenso, e ordenam a gravação de novos blocos. A
comunicação entre nodes é realizada via protocolo GRPC com segurança TLS, e o
protocolo de gossip permite comunicação Peer-to-Peer.

\newpage{}

\paragraph{}
Os Peers têm várias funções, como armazenar cópias do channel (ledger) e
executar contratos inteligentes (chaincodes). Cada peer é identificado por um
certificado digital e representa uma organização específica. Existem tipos
especiais de peers, como o Anchor Peer, que comunica com outras organizações e
nodes de ordenação para manter o channel sincronizado, e o Endorsement Peer, que
armazena o chaincode e processa propostas de transação enviadas pelos clientes
da rede. Para otimizar espaço, os peers podem utilizar snapshots do ledger.

\begin{figure}[H]
	{\includegraphics[scale=0.75,page=5]{\pdf}}
	\caption{Anatomia do Peer}
	%\note{}
	\label{f:figure4}\end{figure}

\paragraph{}
O Orderer é um componente crítico no Hyperledger Fabric, responsável por
garantir o consenso e o permissionamento da Business Network. Ele armazena
channels, identifica permissões e valida características dos chaincodes,
incluindo políticas de endosso. O Orderer recebe pacotes de transações
assinadas, valida-os, organiza as transações, gera novos blocos, e os envia para
os Anchor Peers, mantendo a ordem e consistência da rede.

\paragraph{}
A Certification Authority (CA) gerencia o permissionamento da rede,
representando organizações específicas por meio de uma infraestrutura PKI que
utiliza certificados digitais x.509. Estes certificados são usados pelo
Membership Service Provider (MSP), que abstrai o permissionamento e organiza os
elementos da rede como peers, orderers, e usuários, assegurando a integridade da
rede.

\newpage{}

\paragraph{}
Os Chaincodes são programas que realizam propostas de alterações nos assets, mas
não atualizam diretamente o ledger. Cada chaincode possui uma política de
endosso que define quais assinaturas são necessárias para validar uma transação.
A Endorsing Policy é uma regra lógica que pode exigir assinaturas de múltiplas
organizações e é aplicada para garantir a conformidade das transações.

\paragraph{}
O Client é a interface que conecta o ambiente digital externo ao blockchain,
comunicando-se com peers e orderers. Ele cria propostas de transação, recebe
respostas de execução dos chaincodes e utiliza certificados digitais para
autenticação. Para facilitar a comunicação, os clients utilizam um Connection
Profile, um documento que descreve a configuração da rede.

\paragraph{}
O conceito de Private Data Collection permite que certas informações dentro de
um channel sejam acessíveis apenas a organizações específicas, mantendo dados
sensíveis fora do ledger principal em um banco de dados transiente, como
CouchDB. Os participantes de uma Business Network podem interagir utilizando os
diversos componentes do Hyperledger Fabric, como channels, CAs, orderers, e
clients, garantindo segurança e flexibilidade nas transações empresariais.

% \bibliography{bibliography}

\end{document}
