% This template is based on Pascal Michaillat Minimalist LaTeX Template for
% Academic Papers: https://github.com/pmichaillat/latex-paper
\documentclass[letterpaper,11pt,leqno]{article}
\usepackage{summary}
\usepackage{float}
\usepackage{cite}
\usepackage{siunitx}
\bibliographystyle{bibliography}

% Paper title to populate PDF metadata:
\hypersetup{pdftitle={Resumo: Blockchain GoLedger Módulo 1}}

% Path to BibTeX file with references:
\newcommand{\bib}{bibliography}

% Path to PDF file with figures:
\newcommand{\pdf}{figures/figures}

% To use double quotes
\newcommand{\quotes}[1]{``#1''}

\begin{document}

\begin{figure}[t]
	{\includegraphics[scale=0.25,page=1]{\pdf}}
	\vspace{-50pt}
	\label{f:logo}\end{figure}

% Title:
\title{Resumo: Blockchain GoLedger Módulo 2}

% Authors:
\author{Gabriel dos Santos Schmitz
	%
	% Affiliations and acknowledgements:
	\thanks{Gabriel dos Santos Schmitz: UTFPR. Eu agradeço colegas e orientador
		por comentários e discussões úteis. Este trabalho foi apoiado por Grupo PQC
		UTFPR}}

% Date:
\date{Novembro 2024}

% Permanent URL:
\available{https://github.com/gabrielzschmitz/research/tree/main/summaries/ESR-BlockchainCourse}

\begin{titlepage}

	\maketitle

	% Abstract:
	This document explores public ledgers and Distributed Ledger Technology (DLT) as
	the foundation for blockchain, a decentralized system where transactions are
	securely recorded without central authority. Public ledgers allow transparent
	transaction tracking, while DLT distributes ledger copies across multiple
	participants to establish trust. Consensus protocols like Proof of Work (PoW),
	Proof of Stake (PoS), and Proof of Authority (PoA) maintain data integrity by
	validating blocks, ensuring a secure record in blockchain networks. Privacy
	measures, such as separate chains and off-chain storage, protect sensitive
	data, while tools like Zero Knowledge Proofs (ZKP) enhance security without
	compromising privacy. Blockchain elements—wallets, nodes, and
	oracles—facilitate its functionality, while the Byzantine General’s Problem
	illustrates consensus challenges in unreliable networks.

\end{titlepage}

% Topic 1
\section{Ledgers Públicos}

\paragraph{}
Um ledger público é uma lista de transações acessível a todas as partes
interessadas, que pode ser estruturado de forma permissionada (com controle
sobre quem pode escrever) ou não-permissionada (aberta para qualquer
participante). Exemplos incluem uma conta-corrente bancária, que funciona como
um ledger privado, e o placar de um jogo de futebol, que é um ledger público.

\paragraph{}
Em um ledger público, pessoas como Bob, Alice, Charlie e outros podem registrar
e verificar transações, garantindo uma ordem cronológica. Neste cenário,
qualquer um pode adicionar uma linha, o que torna o processo de registro mais
transparente, mas também exige mecanismos para evitar alterações fraudulentas.

\paragraph{}
Para evitar que uma pessoa, como Bob, escreva transações em nome de outra, como
Alice, são necessários métodos de autenticação que confirmem a identidade de
quem realiza cada transação, prevenindo tentativas de fraude e garantindo a
integridade das entradas.

\paragraph{}
Para autenticar transações, utiliza-se uma chave pública (pk) e um segredo (sk)
para cada participante. As funções Sign e Verify trabalham juntas: Sign cria uma
assinatura com base no conteúdo e no segredo da pessoa, e Verify valida essa
assinatura com a chave pública correspondente, resultando em uma verificação
boolean.

\paragraph{}
Já para evitar duplicidade, como se Bob tentasse copiar e colar uma linha já
assinada, é adicionado um identificador único para cada linha. Esse ID impede
que transações sejam repetidas sem um novo registro único, reforçando a
autenticidade e originalidade das transações no ledger. Isto ocorre pois o ID é
usado na geração do hash da transação.

% Topic 2
\section{DLTs}

\paragraph{}
Para que todos possam acessar e confiar no ledger, é essencial que exista
confiança entre as partes. A descentralização é facilitada pelo compartilhamento
de mensagens entre os participantes, como Alice, Bob, Charlie e outros, para que
cada um receba uma cópia atualizada do ledger.

\paragraph{}
A Distributed Ledger Technology (DLT) permite que múltiplas cópias de um ledger
sejam distribuídas entre os participantes. Esse conceito de registros
distribuídos remonta a práticas antigas, como no sistema financeiro do Império
Romano, onde informações eram compartilhadas de forma descentralizada.

\paragraph{}
Para garantir a segurança no envio de informações entre partes e mitigar falhas
e ataques, é necessário um protocolo de consenso. Esse protocolo ajuda a
sincronizar as mensagens e resolver problemas de consistência no ledger
distribuído, garantindo que as informações sejam confiáveis e atualizadas.

\paragraph{}
Consenso é o processo de decisão coletiva entre as partes. Por exemplo, um grupo
pode decidir um caminho usando a maioria simples ou um método em que alguns
votos (como de personagens azuis) têm mais peso. Consenso permite que os
participantes concordem em uma decisão comum, mesmo em cenários complexos.

\begin{figure}[H]
	{\includegraphics[scale=0.5,page=2]{\pdf}}
	\caption{Exemplo de Consenso}
	%\note{}
	\label{f:figure1}\end{figure}

\paragraph{}
Mensagens e transações são agrupadas em blocos, com a gravação do próximo bloco
sendo decidida pelas regras de consenso. Esse método de consenso vincula tempo e
esforço a registros confiáveis, conectando cada bloco ao anterior por meio de
hashes. Assim, qualquer alteração em um bloco altera os hashes subsequentes,
protegendo a integridade da cadeia.

\paragraph{}
DLT, consenso e blocos conectados formam o protocolo de confiança digital, que
possibilita a existência de um ledger digital público e distribuído. Nesse
sistema, as transações são compartilhadas e assinadas pelos participantes, e um
protocolo de consenso é utilizado para resolver disputas, garantindo a
integridade e confiança no registro das operações.

\begin{figure}[H]
	{\includegraphics[scale=1.0,page=3]{\pdf}}
	\caption{DLT + Consenso + Blocos Conectados}
	%\note{}
	\label{f:figure2}\end{figure}

\section{Conceitos Blockchain}

\paragraph{}
Blockchain é uma tecnologia de ledger distribuído (DLT) com redundância
peer-to-peer, onde as transações são assinadas digitalmente e armazenadas em
blocos. Esses blocos são encadeados por meio do hash do bloco anterior, formando
uma sequência cronológica. As transações podem conter dados ou programas, e um
protocolo de consenso é usado para registrar permanentemente cada bloco na
rede.

\paragraph{}
Os principais componentes de um blockchain incluem transações, nós (nodes),
blocos, o ledger, estado atual (state), contratos inteligentes (smart
contracts), rede peer-to-peer, consenso e oráculos. Esses elementos interagem
para manter a integridade, segurança e funcionalidade de toda a rede.

\paragraph{}
Wallets são plataformas ou dispositivos que gerenciam pares de chaves,
identificando pessoas ou entidades que realizam transações na rede. Elas
armazenam as chaves privadas e podem operar com criptomoedas ou em redes
permissionadas. A segurança pode incluir assinaturas de hardware, autenticação
em dois fatores e biometria, além de apresentar o histórico de transações e o
saldo de criptomoedas.

\paragraph{}
Hierarchical deterministic keys (chaves determinísticas hierárquicas) são
geradas por meio de mnemônicos — palavras escolhidas de um conjunto finito —,
conforme a proposta BIP-0039. Essas palavras permitem a criação de chaves
previsíveis, garantindo consistência e segurança nas transações.

\paragraph{}
Nodes são dispositivos que executam o software de blockchain, formando uma rede
peer-to-peer distribuída. Eles se conectam com outros nodes e participam da
validação de blocos e do algoritmo de consenso. Existem diferentes tipos de
nodes: validadores (mineradores ou notários), archive nodes (que armazenam todo
o histórico de dados), full nodes (para validação e armazenamento) e light nodes
(que apenas armazenam cabeçalhos de blocos).

\paragraph{}
Oráculos são fontes externas de dados que alimentam o blockchain com informações
do mundo real, criando uma ponte entre sistemas externos e o ledger distribuído.
Os dados de oráculos precisam ser convertidos em ativos no blockchain, com
fontes variando desde APIs de clima até dispositivos IoT para contagem de
produtos.

\paragraph{}
O World State é o registro do último estado válido do blockchain, refletindo as
mudanças mais recentes. Cada blockchain permissionado usa um banco de dados para
armazenar seu World State. Por exemplo, se João vende uma BMW para Mauro, o
World State atualizaria para indicar Mauro como o novo proprietário da BMW.

\section{Privacidade Blockchain}

\paragraph{}
A separação em múltiplas chains permite a criação de diferentes ledgers para
aumentar a privacidade em uma rede blockchain. Cada ledger pode ter controle de
acesso independente, limitando quem pode visualizá-lo. A segurança pode ser
garantida pelo monitoramento das operações de leitura registradas, permitindo um
controle mais restrito e personalizado sobre os dados.

\paragraph{}
Algumas blockchains suportam o conceito de dados privados, armazenando
informações confidenciais em bancos de dados transientes, localizados fora do
ledger principal. Essa integração off-chain permite que apenas partes
autorizadas acessem dados específicos, protegendo a privacidade sem comprometer
a segurança do sistema blockchain.

\begin{figure}[H]
	{\includegraphics[scale=0.75,page=4]{\pdf}}
	\caption{Off-chain integrada}
	%\note{}
	\label{f:figure3}\end{figure}

\paragraph{}
Zero Knowledge Proof (ZKP) é um método que permite provar o conhecimento de uma
informação sem revelá-la. Em um exemplo ilustrativo, Alice, que é daltônica,
quer garantir que duas luvas sejam da mesma cor. Bob, que conhece as cores,
responde a Alice se as luvas são iguais, sem revelar a cor. Alice repete o teste
várias vezes; a probabilidade de Bob acertar por sorte diminui exponencialmente
com o número de repetições (${(1/2)}^n$), tornando a resposta de Bob confiável.

\section{Consenso}

\paragraph{}
O consenso em blockchain inclui mecanismos para assegurar a integridade das
transações. Em blockchains não-permissionados, não há garantia de que os nós são
confiáveis. Já em modelos permissionados, os participantes são identificados,
permitindo controle sobre quem pode propor e validar transações, aumentando a
segurança contra ações maliciosas.

\paragraph{}
Problema do General Bizantino, dilema matemático, formulado nos anos 80, ilustra
as dificuldades de coordenação em redes com comunicações não confiáveis. Imagine
batalhões cercando um castelo que precisam atacar ou recuar simultaneamente para
o sucesso. No entanto, as mensagens entre eles podem ser atrasadas, corrompidas,
reordenadas ou interceptadas pelo inimigo, complicando a coordenação e
destacando a necessidade de consenso confiável para garantir ações unificadas.

\paragraph{}
O PoW foi o primeiro mecanismo de consenso blockchain, projetado para evitar o
\quotes{double spending} (gasto duplo). Nele, mineradores competem para resolver
um problema de hash, e o bloco minerado é validado pela rede. Esse processo é
não-determinístico e resiliente ao problema bizantino, embora tenha desafios
como a reorganização de cadeias (forks) e ausência de penalidades diretas para
mineradores mal-intencionados.

\paragraph{}
No PoS, validadores com criptomoedas como colateral competem para registrar
novos blocos. Um validador é escolhido aleatoriamente, enquanto os outros
verificam a transação. Se houver fraude, o validador é penalizado com a perda do
colateral. Esse sistema é bizantinamente tolerante a falhas (BFT), oferecendo
uma alternativa mais eficiente ao PoW.

\paragraph{}
No PoA, validadores autorizados e identificados registram as transações e
validam blocos. Novos validadores precisam ser aprovados para entrar na rede.
Esse método pode ser BFT ou não, dependendo das configurações, e é utilizado em
redes onde a identidade e a autoridade dos validadores são importantes para o
funcionamento seguro e eficiente da blockchain.

% \bibliography{bibliography}

\end{document}
