% This template is based on Pascal Michaillat Minimalist LaTeX Template for
% Academic Papers: https://github.com/pmichaillat/latex-paper
\documentclass[letterpaper,11pt,leqno]{article}
\usepackage{summary}
\usepackage{float}
\usepackage{cite}
\usepackage{siunitx}
\bibliographystyle{bibliography}

% Paper title to populate PDF metadata:
\hypersetup{pdftitle={Resumo: Blockchain GoLedger Módulo 4}}

% Path to BibTeX file with references:
\newcommand{\bib}{bibliography}

% Path to PDF file with figures:
\newcommand{\pdf}{figures/figures}

% To use double quotes
\newcommand{\quotes}[1]{``#1''}

% Adding code support
\usepackage{listings}
\usepackage{xcolor}
\definecolor{darkBackground}{RGB}{40, 44, 52} % Background color
\definecolor{lightGray}{RGB}{55, 61, 69}      % Light gray for the frame
\definecolor{blue}{RGB}{66, 134, 244}         % Blue for keywords
\definecolor{green}{RGB}{80, 250, 123}        % Green for comments
\definecolor{red}{RGB}{255, 85, 85}           % Red for strings
\definecolor{purple}{RGB}{189, 147, 249}      % Purple for other keywords
\definecolor{orange}{RGB}{255, 184, 108}      % Orange for constants
\lstdefinestyle{oneDarkStyle}{
    backgroundcolor=\color{darkBackground}, % Solid background
    basicstyle=\ttfamily\footnotesize\color{white}, % Font style
    breaklines=false,
    frame=single,
    captionpos=b,
    rulecolor=\color{darkBackground}, % Change frame color to match background
    keywordstyle=\color{blue}\bfseries,
    commentstyle=\color{green},
    stringstyle=\color{red},
    numberstyle=\tiny\color{black},
    showtabs=false,                  
    numbers=left,
    morekeywords={int, float, return, void}, % Extra keywords
}
\lstset{style=oneDarkStyle}

\begin{document}

\begin{figure}[t]
	{\includegraphics[scale=0.25,page=1]{\pdf}}
	\vspace{-50pt}
	\label{f:logo}\end{figure}

% Title:
\title{Resumo: Blockchain GoLedger Módulo 4}

% Authors:
\author{Gabriel dos Santos Schmitz
	%
	% Affiliations and acknowledgements:
	\thanks{Gabriel dos Santos Schmitz: UTFPR. Eu agradeço colegas e orientador
		por comentários e discussões úteis. Este trabalho foi apoiado por Grupo PQC
		UTFPR}}

% Date:
\date{Novembro 2024}

% Permanent URL:
\available{https://github.com/gabrielzschmitz/research/tree/main/summaries/ESR-BlockchainCourse}

\begin{titlepage}

	\maketitle

	% Abstract:
	This article explores key concepts of Hyperledger Fabric, focusing on
	transaction flow, private data handling, and network characteristics. It
	begins with the transaction process, where a client initiates a proposal,
	which is validated by endorsing peers before being organized and added to the
	ledger by the Ordering Service. This flow is divided into three stages:
	proposal and execution, submission and ordering, and validation and
	commitment. The article also covers specific scenarios like managing Private
	Data Collections (PDC) for sensitive data, highlighting how PDC adds
	complexity to transaction flow while maintaining confidentiality.
	Additionally, network scalability is discussed, addressing the adaptation to
	organizational changes and the role of Peer Snapshots in data efficiency.
	Techniques for chaincode interoperability, transaction tuning, and monitoring
	for malicious activities underscore the flexibility and robustness of
	Hyperledger Fabric. The piece concludes with a mention of future improvements,
	particularly the planned Byzantine Fault Tolerant (BFT) consensus in Fabric
	3.0, enhancing reliability and security in decentralized environments.

\end{titlepage}

% Topic 1
\section{Conceitos de Hyperledger Fabric II}

\begin{figure}[H]
	{\includegraphics[scale=0.5,page=2]{\pdf}}
	\caption{Transaction flow}
	%\note{}
	\label{f:figure1}\end{figure}

\textbf{1. Client Creates and Sends a Transaction:} The client creates a
\textbf{PROPOSE MESSAGE} and sends it to a peer with the Gateway Service, which
forwards it to the necessary endorsing peers according to the chaincode's
endorsing policy. The \textbf{PROPOSE MESSAGE} includes:
\begin{itemize}
	\item clientID
	\item chaincodeID
	\item txPayload
	\item Timestamp
	\item clientSig
\end{itemize}

\textbf{2. Endorsing Peers Simulate and Sign the Transaction:} The endorsing
peers execute the chaincode using the information from the \textbf{PROPOSE
	MESSAGE}. The result is digitally signed, generating a signed transaction, which
is sent back to the peer with the Gateway Service.

\textbf{3. Client Sends Signed Transactions to the Ordering Service:} The client
receives the signed transactions from the endorsing peers via the Gateway
Service and creates an \textbf{ENDORSEMENT} package. This package is then sent
to the Ordering Service.

\textbf{4. Ordering Service Delivers the Block to Peers:} The Ordering Service
validates the transactions based on the chaincode's endorsing policy. If all
required endorsing peers have signed the transaction, it is delivered as a block
to all peers, updating their state.

\newpage

\textbf{5. Peers Validate Transactions:} Each peer validates the consistency of
the read/write sets of the transactions. Inconsistent transactions are marked as
invalid, while valid transactions update the state.

\begin{figure}[H]
	{\includegraphics[scale=0.5,page=3]{\pdf}}
	\caption{Fluxo de uma transação Hyperledger Fabric}
	%\note{}
	\label{f:figure2}\end{figure}

\subsection{Exemplo de Fluxo}

\paragraph{}
A Organização A compra e vende legumes da Organização B, e o consenso do
chaincode estabelece que ambas as partes devem assinar as transações. Tanto A
quanto B possuem um peer para enviar pedidos de compra, aceite ou rejeição. O
processo começa quando o cliente da Org A inicia a transação usando um SDK para
enviar uma proposta de transação aos endorsing peers de A e B.

\paragraph{}
Os endorsing peers recebem a proposta de transação, verificam sua autenticidade
(incluindo formato, não duplicação, e autorização do cliente), e executam o
chaincode. Se tudo estiver correto, o resultado é enviado de volta ao Gateway. O
cliente coleta essas respostas dos endorsing peers e aguarda o próximo passo.

\paragraph{}
O Gateway de A monta a transação de endorsement e envia para o Ordering Service,
que valida as transações de acordo com a política de endosso e organiza-as
cronologicamente em blocos. Depois disso, os blocos são adicionados ao canal.

\paragraph{}
Os peers verificam as transações no bloco e atualizam o ledger. Um evento é
então enviado ao cliente A, informando que a transação foi registrada com
sucesso. Esse processo atualiza o estado global (World State) do ledger.

\paragraph{}
O fluxo é dividido em três fases:

\textbf{Proposta e Execução:} A proposta é feita, o chaincode é executado, e o
conjunto de leitura/escrita (R/W Set) é assinado pelos peers.
\newline
\textbf{Envio e Ordenação:} A transação é enviada ao Orderer para verificação e ordenação.
\newline
\textbf{Validação e Comprometimento:} As transações são validadas quanto à
consistência, o estado é atualizado, e o cliente é notificado.

\paragraph{}
Se o chaincode usa Private Data Collections (PDC), o fluxo inclui uma etapa
adicional para gerar uma resposta de proposta de transação. Dados privados são
armazenados separadamente no CouchDB, e replicados apenas entre os peers
autorizados pelas políticas de PDC. O uso de dados transitórios permite que
argumentos sensíveis não sejam gravados nos blocos.

\begin{figure}[H]
	{\includegraphics[scale=0.5,page=4]{\pdf}}
	\caption{Private Data Collections}
	%\note{}
	\label{f:figure3}\end{figure}

\newpage
\paragraph{}
Para iniciar uma transação, o SDK do Hyperledger precisa de informações sobre a
rede, como políticas de endosso, peers e orderers de um canal. O Service
Discovery fornece essas informações de forma dinâmica, complementando o perfil
de conexão da rede.

\begin{figure}[H]
	{\includegraphics[scale=0.5,page=5]{\pdf}}
	\caption{Service Discovery}
	%\note{}
	\label{f:figure4}\end{figure}

\subsection{Características Redes Fabric}

\paragraph{}
Para manter a rede escalável, alguns eventos podem exigir mudanças na
configuração, como a adição de novas organizações ou nós, revogação de
certificados (CRL) para excluir entidades, mudanças na hierarquia de consenso e
atualizações de contratos inteligentes. Essas adaptações garantem que a rede
possa crescer e se ajustar a novas necessidades sem comprometer a segurança ou
funcionalidade.

\paragraph{}
O recurso de Peer Snapshot permite reduzir a quantidade de dados armazenados no
ledger de um peer, economizando espaço e diminuindo o tempo de sincronização de
novos peers. Por exemplo, um peer pode começar a partir do bloco \#1000,
ignorando dados anteriores. No entanto, isso limita as consultas ao histórico e
não suporta dados privados (private-data).

\paragraph{}
Um chaincode pode interagir com outro utilizando o método invokeChaincode.
Quando a chamada ocorre no mesmo canal, o read/write set do chaincode chamado é
integrado à resposta principal. Se a chamada é feita entre canais diferentes,
apenas a resposta é retornada, sem alterar a proposta de transação original.

\paragraph{}
A otimização da velocidade de registro depende de ajustes em várias áreas, como
a política de endosso, métodos específicos do chaincode, parâmetros do cliente e
uso de Private Data Collections (PDC). Essas configurações permitem balancear
segurança e desempenho conforme as necessidades do sistema.

\paragraph{}
Embora o consenso do Hyperledger Fabric não seja Byzantine Fault Tolerant (BFT),
existem mecanismos para detectar comportamentos suspeitos. Clientes podem
comparar as respostas de propostas de transação, e peers que enviam respostas
inconsistentes podem ser tratados de forma diferenciada. Em cenários de consenso
democrático, peers anômalos podem ser sinalizados para administração, que decide
sobre permissões e atualizações. O suporte a BFT está previsto para a versão 3.0
do Fabric.

% \bibliography{bibliography}

\end{document}
