% This template is based on Pascal Michaillat Minimalist LaTeX Template for
% Academic Papers: https://github.com/pmichaillat/latex-paper
\documentclass[letterpaper,11pt,leqno]{article}
\usepackage{summary}
\usepackage{float}
\usepackage{cite}
\usepackage{siunitx}
\bibliographystyle{bibliography}

% Paper title to populate PDF metadata:
\hypersetup{pdftitle={Resumo: Blockchain GoLedger Módulo 1}}

% Path to BibTeX file with references:
\newcommand{\bib}{bibliography}

% Path to PDF file with figures:
\newcommand{\pdf}{figures/figures}

% To use double quotes
\newcommand{\quotes}[1]{``#1''}

\begin{document}

\begin{figure}[t]
	{\includegraphics[scale=0.25,page=1]{\pdf}}
	\vspace{-50pt}
	\label{f:figure1}\end{figure}

% Title:
\title{Resumo: Blockchain GoLedger Módulo 1}

% Authors:
\author{Gabriel dos Santos Schmitz
	%
	% Affiliations and acknowledgements:
	\thanks{Gabriel dos Santos Schmitz: UTFPR. Eu agradeço colegas e orientador
		por comentários e discussões úteis. Este trabalho foi apoiado por Grupo PQC
		UTFPR}}

% Date:
\date{Outubro 2024}

% Permanent URL:
\available{https://github.com/gabrielzschmitz/research/tree/main/summaries/ESR-BlockchainCourse}

\begin{titlepage}

	\maketitle

	% Abstract:
	This document delves into the evolution of value exchange and the emergence of
	blockchain technology, focusing on Bitcoin as a decentralized digital currency
	that revolutionized traditional finance. By eliminating the need for
	intermediaries, blockchain allows secure peer-to-peer transactions through
	cryptographic principles, using structures like UTXOs and Merkle trees to
	ensure integrity. Bitcoin’s Proof of Work (PoW) mechanism underpins its
	consensus model, though it has significant environmental costs, prompting some
	blockchains to adopt alternatives like Ethereum’s Proof of Stake (PoS). Since
	its inception, Bitcoin has evolved from a niche asset to a widely recognized
	store of value and payment method, even gaining status as legal tender in some
	countries.

\end{titlepage}

% Topic 1
\section{Origem e Usos de Blockchain}

\paragraph{}
A atribuição de valor a objetos é uma atividade subjetiva e marginal, ou seja,
dependente das circunstâncias e da percepção individual. Cada pessoa confere
valor de forma particular, mesmo que influenciada pela opinião de outros.
Historicamente, as primeiras transações comerciais foram realizadas por meio do
escambo, onde um objeto A era trocado por um objeto B, sendo que B era mais
valioso para um lado da transação e A para o outro. Essa forma de troca,
entretanto, apresenta limitações: se um indivíduo possui grande quantidade de um
item que não lhe interessa, o escambo pode se tornar ineficaz ou inviável.
Surge, então, a necessidade de um item com valor de troca \textit{universal} ---
a moeda.

\paragraph{}
O primeiro tipo de moeda adotado foi o gado bovino, na Grécia, seguido pelo uso
de sal como moeda de troca, o que originou a palavra \quotes{salário} que
utilizamos até hoje. Para ilustrar uma moeda arcaica, o curso utiliza as pedras
Rai das Ilhas Yap, que variavam de tamanho entre 3 \unit{\centi\metre} e 3
\unit{\metre}. Essas pedras possuíam um valor associado tanto ao seu tamanho
quanto à sua história, conferindo ao proprietário status e riqueza. Em uma
analogia mais moderna, temos os NFTs (Tokens Não Fungíveis), que possuem valor
artístico e são valorizados pela sua unicidade, assegurada por um \textit{hash}
que garante a autenticidade e indivisibilidade do item, de forma semelhante a
uma obra de arte única.

\paragraph{}
Assim, a tecnologia de blockchain oferece um meio seguro e verificável de
estabelecer e manter a autenticidade e a propriedade de ativos digitais, de
forma descentralizada, com potencial para diversos usos além das transações
financeiras, como contratos inteligentes e autenticação digital.

\begin{figure}[H]
	{\includegraphics[scale=1.5,page=2]{\pdf}}
	\caption{Pedras Rai e NFT}
	%\note{}
	\label{f:figure1}\end{figure}

% Topic 2
\section{Do Livro Razão ao Blockchain}

\paragraph{}
Os primeiros livros-razão surgiram por volta de \num{5000} a.C., quando as
sociedades passaram a registrar seus bens e transações. Antes disso, as pessoas
mantinham apenas o que conseguiam carregar, limitando o potencial de acumulação
e troca de recursos. Com o avanço das trocas comerciais, esses registros se
tornaram essenciais para documentar a posse de bens e acordos de troca. Na
atualidade, o \textit{ledger} digital, popularizado pelo advento do Bitcoin,
representa uma evolução significativa desses antigos livros-razão, funcionando
como um registro de todas as transações da rede de forma cronológica e linear,
sem a necessidade de uma entidade centralizadora.

\paragraph{}
Inicialmente, os livros-razão tradicionais tinham um formato simples, com uma
única entrada por linha. Com o tempo e o aumento da complexidade das transações,
tornaram-se mais detalhados, passando a incluir a história das transações de
ambas as partes envolvidas, além dos registros de débitos e créditos, que
refletiam de forma mais precisa o fluxo de ativos.

\paragraph{}
No contexto do Bitcoin e das criptomoedas, o \textit{ledger} digital, conhecido
como \textit{blockchain}, possui o papel fundamental de manter a transparência e
a confiança na rede. Esse livro-razão digital requer que todas as transações
sejam registradas em sequência, desde a criação do primeiro bloco, garantindo a
integridade e a imutabilidade dos dados. Dessa forma, o blockchain opera como
uma estrutura descentralizada que possibilita o armazenamento seguro e público
de informações, tornando-se um elemento central para diversas aplicações,
incluindo contratos inteligentes e autenticações digitais.

% Topic 3
\section{Fidúcia e o Papel dos Intermediários}

\paragraph{}
Em uma transação de troca de produtos, valores ou serviços, certos elementos
fundamentais garantem que o processo seja justo e satisfatório para ambas as
partes. Primeiramente, cada participante deve avaliar o valor que atribui ao
item trocado, pois essa percepção de valor assegura que ambas as partes se
sintam beneficiadas com a transação. A confiança (ou fidúcia) entre as partes é
igualmente crucial, pois transações comerciais dependem de uma relação de
credibilidade que evita a necessidade de intervenções externas. Em muitos casos,
são exigidas garantias adicionais para assegurar que cada lado cumpra suas
obrigações, promovendo assim uma troca mais segura e confiável.

\paragraph{}
Conforme argumentado por Aristóteles, esses elementos são ainda mais eficientes
quando potencializados pela introdução da moeda, um meio de troca que substitui
o escambo direto e facilita o comércio. Segundo o filósofo, a moeda cumpre três
funções essenciais: age como meio de troca, permitindo transações mais ágeis;
como unidade de conta, permitindo a comparação objetiva entre valores de
diferentes bens e serviços; e como reserva de valor, possibilitando o acúmulo de
riqueza ao longo do tempo. Para que a moeda cumpra bem seu papel, deve possuir
características específicas, como durabilidade, divisibilidade,
transportabilidade e valor intrínseco. O registro adequado das transações, feito
em sistemas como livros-razão ou em tecnologias mais modernas como o blockchain,
contribui para a transparência e rastreabilidade, completando o ciclo de troca
com segurança.

\paragraph{}
A necessidade de garantias em uma transação está diretamente ligada ao nível de
confiança entre as partes. Quanto maior a confiança, menor a exigência de
garantias, pois cada parte tem mais certeza do cumprimento do acordo. Quando a
confiança é baixa, as garantias devem ser maiores, sendo comum recorrer a um
intermediário para gerenciar essas garantias e formalizar o processo. Esse
intermediário precisa ser confiável para ambas as partes e ter a
responsabilidade de registrar a transação. Exemplos de intermediários incluem
cartórios, que gerenciam registros para diversos setores; a Justiça, que atua
como mediadora de disputas; a Previdência, para o gerenciamento de contribuições
e aposentadorias; e os bancos, que, no sistema financeiro, gerenciam e guardam
valores monetários.

\paragraph{}
A história dos sistemas de valor reflete uma evolução constante, desde o uso de
pedras e conchas até metais preciosos que serviram como lastro para as moedas
emitidas por governos. O fim do sistema Bretton Woods trouxe uma mudança
significativa, substituindo moedas lastreadas em ativos por moedas fiduciárias,
cujo valor é baseado na confiança dos usuários no emissor — geralmente um
governo — em vez de ativos tangíveis.

\paragraph{}
O sistema financeiro moderno exerce um papel fundamental ao realizar funções
como guarda de valor, registro de transferências e concessão de empréstimos.
Historicamente, esses registros eram centralizados em livros-razão, mas
evoluíram para livros-razão distribuídos (DLT), uma prática que teve origem no
Império Romano e que hoje se beneficia das tecnologias digitais. No entanto, a
governança desses sistemas nem sempre é infalível, o que pode abrir brechas para
fraudes, como evidenciado pela crise do subprime de 2008, que expôs falhas
significativas na gestão de ativos e no controle de crédito no mercado
financeiro.

% Topic 4
\section{Bitcoin}

\paragraph{}
Em novembro de 2008, Satoshi Nakamoto lançou um paper intitulado A Peer-to-Peer
Electronic Cash System, que estabeleceu os fundamentos para o Bitcoin e seu
sistema de transações eletrônicas em uma rede distribuída. O documento descreveu
a criação de um \quotes{ativo} digital cuja propriedade é registrada em um
ledger público (blockchain) e operado em uma rede peer-to-peer. Esse sistema
permite que transações ocorram entre usuários de forma descentralizada, sem a
necessidade de uma autoridade central, resultando na criação do Bitcoin como o
conhecemos hoje.

\paragraph{}
O protocolo do Bitcoin funciona através de um mecanismo de identificação onde
cada usuário é representado por uma chave pública e protegida por uma chave
privada, que funciona como uma senha. Transações são realizadas e enviadas para
a rede, onde aguardam validação. Os computadores da rede, conhecidos como
\quotes{mineradores}, competem para agrupar essas transações em blocos válidos.
Esse processo de validação envolve a resolução de problemas matemáticos
complexos, recompensando o minerador que consegue resolver o problema com o
direito de incluir o próximo bloco na blockchain.

\paragraph{}
A estrutura de blocos no Bitcoin utiliza um encadeamento de hashes para manter a
sequência e a integridade dos registros. Cada bloco contém um hash do bloco
anterior, formando uma cadeia ininterrupta que confirma a sequência das
transações. Essa metodologia possibilita que transações ocorram entre pessoas ou
entidades que não se conhecem ou confiam entre si, em uma rede que também não
oferece confiança intrínseca, mas que, ainda assim, garante segurança e
integridade.

\paragraph{}
Para manter o controle de saldos e prevenir gastos duplos, o Bitcoin adota o
conceito de UTXO (Unspent Transaction Output). Esse sistema armazena o
\quotes{troco} de transações, que são saldos remanescentes disponíveis para uso
em transações futuras. Esse mecanismo permite que o Bitcoin mantenha um controle
eficiente dos saldos sem que seja necessário verificar todo o histórico do
ledger, garantindo a continuidade das operações financeiras na rede de forma
descentralizada e segura.

\begin{figure}[H]
	{\includegraphics[scale=0.5,page=3]{\pdf}}
	\caption{UTXO}
	%\note{}
	\label{f:figure2}\end{figure}

\textbf{Transação Bitcoin}
\begin{itemize}
	\item \textbf{Entrada:}
	      \begin{itemize}
		      \item Conjunto de UTXOs (Unspent Transaction Outputs)
	      \end{itemize}
	\item \textbf{Saída:}
	      \begin{itemize}
		      \item UTXO (novo output gerado para o destinatário)
		      \item Transferência
		      \item \textit{Fee} (taxa para mineradores pela validação da transação)
	      \end{itemize}
\end{itemize}

\begin{figure}[H]
	{\includegraphics[scale=0.5,page=4]{\pdf}}
	\caption{UTXO}
	%\note{}
	\label{f:figure3}\end{figure}

\textbf{Árvore de Merkle}

\textbf{Bloco Bitcoin}
\begin{itemize}
	\item \textbf{Tamanho:}
	      \begin{itemize}
		      \item 1 MB
	      \end{itemize}
	\item \textbf{Estrutura:}
	      \begin{itemize}
		      \item Árvore de Merkle
		      \item Utilizada para garantia de integridade e velocidade
	      \end{itemize}
\end{itemize}

\paragraph{}
O consenso no Bitcoin é garantido pelo mecanismo de Prova de Trabalho (PoW), que
exige um alto gasto de tempo e energia, resultando em uma rede confiável e
resistente a manipulações. Esse consenso é não-determinístico, o que significa
que forks (bifurcações) na blockchain podem ocorrer, levando a reorganizações
para resolver quais transações são válidas. A Prova de Trabalho, embora eficaz
em garantir segurança, tem uma alta pegada de carbono, com um consumo estimado
superior a 50 TWh por ano entre Bitcoin e Ethereum. Cada transação de Bitcoin
consome aproximadamente 1 kWh de energia.

\paragraph{}
Atualmente, novos blockchains estão adotando alternativas mais sustentáveis,
como o Ethereum, que migrou para Prova de Participação (PoS) após o Merge,
visando reduzir o impacto ambiental. A rede Bitcoin, no entanto, continua
acessível para qualquer máquina participar e qualquer usuário realizar
transações, garantindo a descentralização e a abertura da rede através de pares
de chaves gerados aleatoriamente com o algoritmo ECDSA.\@

\paragraph{}
Satoshi Nakamoto revolucionou ao criar um ledger digital público e distribuído
com o Bitcoin, integrando tecnologias como redes peer-to-peer, assinaturas
digitais e o conceito de ativo digital escasso. Além disso, Nakamoto incorporou
um sistema de validação de transações não gastas (UTXO) e usou teoria dos jogos
na mineração para resolver fraudes e garantir consenso.

\paragraph{}
Desde sua criação, o valor do Bitcoin evoluiu drasticamente, passando de uma
moeda com pouco valor, como em 2010, quando 10.000 Bitcoins compraram duas
pizzas, para uma unidade valendo cerca de 15.000 dólares em 2022. Hoje, o
Bitcoin é visto como reserva de valor, opção de investimento, meio de pagamento
e até como moeda nacional em países como El Salvador.

% \bibliography{bibliography}

\end{document}
