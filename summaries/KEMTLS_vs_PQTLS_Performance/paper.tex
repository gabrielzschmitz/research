\documentclass[letterpaper,11pt,leqno]{article}
\usepackage{paper}
\usepackage{float}
\usepackage{cite}
\bibliographystyle{bibliography}

% Paper title to populate PDF metadata:
\hypersetup{pdftitle={KEMTS vs. Post-Quantum TLS: um pequeno resumo}}

% Path to BibTeX file with references:
\newcommand{\bib}{bibliography}

% Path to PDF file with figures:
\newcommand{\pdf}{figures/figures}


\begin{document}

% Title:
\title{Resumo: KEMTLS vs. Post-Quantum TLS: Performance On Embedded Systems}

% Authors:
\author{Gabriel dos Santos Schmitz
	%
	% Affiliations and acknowledgements:
	% \thanks{First Author: First University. Second Author: Second University. We
	% thank colleagues for helpful comments and discussions. This work was supported
	% by a grant [grant number]; another grant [grant number]; and a foundation.}}
	\thanks{Gabriel dos Santos Schmitz: UTFPR. Eu agradeço colegas e orientador
		por comentários e discussões úteis. Este trabalho foi apoiado por Grupo PQC
		UTFPR}}

% Date:
\date{Agosto 2024}

% Permanent URL (can be commented out):
\available{https://github.com/gabrielzschmitz/research/tree/main/summaries/KEMTLS_vs_PQTLS_Performance}

\begin{titlepage}

	\maketitle

	% Abstract:
	TLS is widely used to secure communications across various devices, but
	current public key cryptosystems could become vulnerable to quantum computers.
	To address this, post-quantum cryptography (PQC) is being developed, with
	standards from NIST expected soon. However, some NIST PQC finalists pose
	challenges for small microcontrollers due to their large key sizes or hardware
	requirements.

	KEMTLS is an alternative TLS handshake protocol that uses key encapsulation
	mechanisms (KEMs) for authentication instead of signatures, potentially
	offering better efficiency. This study compares KEMTLS with TLS 1.3 in an
	embedded environment using a Cortex-M4-based platform and the WolfSSL library.
	The comparison includes performance metrics such as runtime, memory usage,
	traffic volume, and code size across different network settings like
	broadband, LTE-M, and Narrowband IoT. Results indicate that KEMTLS can reduce
	handshake times by up to 38%, lower peak memory consumption, and decrease
	traffic volume compared to TLS 1.3.

\end{titlepage}

% Introduction 1
\section{Introdução}

\paragraph{}
Segurança de Camada de Transporte (TLS) utiliza um handshake entre servidores
e clientes para troca e verificação de chaves de criptografia, empregando o
algoritmo Diffie-Hellman com curvas elípticas efêmeras. As chaves públicas são
trocadas e verificadas por certificados assinados por autoridades e instalados
previamente nos dispositivos. Com o avanço da computação quântica, o NIST está
desenvolvendo novos padrões de criptografia pós-quântica (PQ), incluindo
Mecanismos de Encapsulamento de Chaves (KEMs). Os algoritmos híbridos combinam a
troca clássica de chaves elípticas e PQ para maior segurança. Dilithium, um dos
esquemas de assinatura, é lento e aumenta o handshake em 17 kb, enquanto Falcon,
que exige operações de ponto flutuante duplo, é menor mas menos viável em
hardware limitado. O KEMTLS propõe uma alternativa ao usar KEMs para
autenticação, sendo mais eficiente em termos de recursos e reduzindo a base de
código. Com a crescente conexão de dispositivos via IPv6, como visto no
protocolo Matter, a criptografia para dispositivos embarcados enfrentará novos
desafios.

% Introduction 1.1
\subsection{Contribuição}

\paragraph{}
Este trabalho investiga se as vantagens do KEMTLS se aplicam a sistemas
embarcados, comparando-o com o PQTLS em um ambiente embutido. Implementaram-se
KEMTLS e PQTLS com todos os esquemas de assinatura e KEMs finalistas do NIST,
exceto o Classic McEliece devido ao tamanho excessivo das chaves públicas,
ultrapassando a ROM desponível na plataforma. A comparação direta de desempenho
é possível devido à sobreposição significativa de código entre as
implementações. O estudo avalia tempo de execução, uso de memória, tamanho do
código e consumo de largura de banda das implementações em uma plataforma
baseada em Cortex-M4. Os experimentos simulam uma conexão TLS entre um
dispositivo embarcado e um servidor de alto desempenho, utilizando tecnologias
de rede típicas. O trabalho apresenta a criptografia pós-quântica, o processo de
padronização, os protocolos PQTLS e KEMTLS, suas diferenças e os detalhes da
implementação e configuração experimental, culminando nos resultados e
conclusões.

% Background 2
\section{Plano de Fundo}

\paragraph{}
Nesta seção, abordaremos o desenvolvimento da criptografia pós-quântica,
incluindo uma visão geral histórica e o processo de padronização do NIST. Também
detalharemos o impacto da criptografia pós-quântica no TLS 1.3 e o
desenvolvimento do KEMTLS.

% Background 2.1
\subsection{Criptografia Pós Quântica}

\paragraph{}
Em 1994, Peter Shor criou algoritmos quânticos que ameaçam a criptografia
pública atual, incluindo o TLS, que seria comprometido por computadores
quânticos grandes. O projeto de padronização de criptografia pós-quântica (PQC)
do NIST começou em 2017, com mais de 60 propostas, das quais apenas 4 KEMs e 3
algoritmos de assinatura foram finalistas. Entre os KEMs baseados em redes estão
Kyber, Saber e NTRU, enquanto Falcon e Dilithium são baseados em assinaturas de
rede. O Classic McEliece, baseado em códigos, foi descartado do artigo por seu
tamanho de chave grande. Os algoritmos são avaliados em três níveis de
segurança: I, III e V, e o estudo foca apenas no nível I. Falcon é baseado em
NTRU, Dilithium em Module-LWE e Short Integer Solution, e Rainbow é uma
assinatura multivariada variante do UOV, recentemente quebrada por Beullens. A
criptografia pós-quântica é desafiadora para dispositivos embarcados devido às
limitações de hardware, e os testes foram realizados sem reduzir a frequência do
Cortex-M4.

\begin{figure}[H]
	{\includegraphics[scale=0.4,page=2]{\pdf}}
	\caption{Comparação dos finalistas da Rodada 3}
	\note{Comparação dos finalistas da Rodada 3 do NIST PQC no nível de segurança I. Mostramos o tamanho (em bytes) dos dados transmitidos durante um handshake (chave pública, assinatura e ciphertext), dados offline (chaves secretas) e tempos de operação (de [9, 17]) no M4.}
	\label{f:table1}\end{figure}

% Background 2.2
\subsection{TLS Pós Quântico}

\paragraph{}
O TLS 1.3, amplamente utilizado no HTTPS, autentica o servidor para o cliente e
oferece suporte opcional para autenticação mútua. O handshake unilateral
autenticado por certificado utiliza uma troca de chaves Diffie-Hellman (DH)
efêmera, seguida de uma assinatura sobre o handshake para autenticar,
finalizando com autenticação adicional via MAC. Para adaptar o TLS 1.3 ao
cenário pós-quântico, é possível substituir a geração de chaves DH do servidor
pelo encapsulamento de uma chave usando KEMs (Key Encapsulation Mechanisms) e
adotar algoritmos de assinatura pós-quânticos em vez de RSA ou assinaturas
baseadas em curvas elípticas. Mesmo com essas mudanças, o TLS 1.3 mantém sua
eficiência, necessitando apenas de 1.5-RTT (round-trip time).

\begin{figure}[H]
	{\includegraphics[scale=0.4,page=3]{\pdf}}
	\caption{Fluxos de protocolo}
	\note{Diagramas simplificados do fluxo do protocolo: (à esquerda) o
		handshake do TLS 1.3, usando assinaturas para autenticação do servidor; e (à
		direita) o handshake do KEMTLS, usando KEMs para autenticação do servidor.}
	\label{f:figure1}\end{figure}

\paragraph{}
O KEMTLS é uma proposta de handshake TLS pós-quântico que substitui assinaturas
tradicionais por mecanismos de encapsulamento de chave (KEMs), que são menores e
mais eficientes em termos computacionais. No KEMTLS, os certificados contêm
chaves públicas para KEMs, enquanto as assinaturas ainda são usadas para
verificar a cadeia de certificados de forma offline. Isso otimiza o tamanho das
chaves públicas e assinaturas, melhorando a eficiência computacional. Inspirado
pelo OPTLS de Krawczyk e Wee, o KEMTLS evita penalizações de desempenho ao
permitir que o cliente envie sua solicitação na mesma etapa do TLS 1.3, mantendo
o protocolo eficiente com apenas 1,5 RTT.

% Experimental Setup 3
\section{Configuração Experimental}

\paragraph{}
A seção a seguir descreve o ambiente experimental utilizado para obter nossos
resultados. Ambos os protocolos foram avaliados em termos de tempo de handshake,
tempo de execução dos algoritmos, uso máximo de memória, tamanho do código e
tráfego de rede. Os tempos de handshake foram medidos em três ambientes de rede
relevantes para o IoT: internet "broadband", LTE-M e Narrowband-IoT (NB-IoT). As
características desses ambientes são detalhadas na Tabela 2. O ambiente
"broadband" é baseado em tempos realistas de comunicação cliente-nuvem na Europa
Ocidental, enquanto LTE-M e NB-IoT têm suas características baseadas em números
do 3GPP.

\begin{figure}[H]
	{\includegraphics[scale=0.4,page=4]{\pdf}}
	\caption{Características de conexão de acordo com 3GPP [1]}
	\label{f:table2}\end{figure}

\paragraph{}
Como o KEMTLS é um protocolo pós-quântico, os experimentos não consideraram
certificados mistos ou métodos híbridos (pós-quântico e curva elíptica). Tanto
KEMTLS quanto PQTLS utilizam uma autoridade certificadora (CA) para assinar
certificados, mas diferem no tipo de chave pública incluída nos certificados
folha: KEMs em KEMTLS e chaves de assinatura em PQTLS. Foram avaliadas
combinações dos finalistas do NIST, exceto Classic McEliece devido ao tamanho
excessivo de suas chaves públicas. Apenas os parâmetros do nível de segurança
mais baixo, o nível I do NIST, foram utilizados por serem mais compactos e
eficientes.

% Experimental Setup 3.1
\subsection{Implementação}

\paragraph{}
Todos os benchmarks foram realizados em uma placa Silicon Labs STK3701A,
conhecida como "Giant Gecko", equipada com um processador ARM Cortex-M4F de 72
MHz e memória suficiente para suportar as chaves públicas do Rainbow. O
Cortex-M4 é a plataforma de referência do NIST para dispositivos embarcados, com
implementações otimizadas para a maioria dos algoritmos finalistas fornecidas
pelo projeto PQM4. As implementações de criptografia pós-quântica (PQC)
utilizadas foram retiradas do projeto PQM4, com pequenas modificações, e
compiladas usando GCC 11.1 com otimização de velocidade.

\paragraph{}
O Giant Gecko atuou como cliente (KEM)TLS, conectado a um servidor backend, com
a comunicação feita via Ethernet. O servidor simulava diferentes ambientes de
rede usando o framework netem do Linux. O sistema usou o Zephyr RTOS, um sistema
operacional em tempo real de código aberto com suporte para TCP/IP e TLS. A
implementação do servidor KEMTLS foi baseada no código de Wiggers, e o WolfSSL
foi escolhido pela sua eficiência de memória e suporte para TLS 1.3. A
integração de algoritmos pós-quânticos no WolfSSL foi simplificada pela API
clara do WolfCrypt, com modificações para suportar chaves KEM.

\paragraph{}
Em ambos os experimentos, PQTLS e KEMTLS, o cliente realizou apenas a
verificação de assinaturas, sem necessidade de código de assinatura. Os
resultados dos benchmarks foram recebidos via comunicação serial, e todos os
experimentos utilizaram a cifra TLS 1.3 TLS\_CHACHA20\_POLY1305\_SHA256.

% Results 4
\section{Resultados}

\paragraph{}
O texto discute os trade-offs importantes para desenvolvedores de sistemas
embarcados entre tamanho de código, uso de memória, tráfego de rede e tempo de
CPU ao utilizar KEMTLS e TLS 1.3 com finalistas do NIST PQC. O tempo de execução
dos algoritmos afeta o consumo de energia, especialmente em dispositivos
alimentados por bateria. O tráfego de rede também impacta o consumo de energia e
pode ser caro dependendo da tecnologia sem fio. Os benchmarks apresentados focam
em plataformas Cortex-M4 e incluem resultados de tamanho de código, uso de
memória, tráfego e duração do handshake, e tempo de execução das primitivas PQC.
Rainbow, um dos finalistas do NIST, está incluído apenas nos resultados do
KEMTLS devido ao grande tamanho de suas chaves públicas. Todos os algoritmos PQC
utilizados foram otimizados para velocidade.

% Results 4.1
\subsection{Armazenamento e Consumo de Memória}

\paragraph{}
As implementações dos protocolos KEMTLS e TLS 1.3 têm tamanhos de código
similares, ambos com aproximadamente 111 kB sem as primitivas pós-quânticas. A
Tabela 3 apresenta as combinações de algoritmos PQC e seus respectivos tamanhos
de código. KEMTLS é mostrado com uma única instância de KEM para troca de chaves
efêmeras e autenticação, enquanto TLS 1.3 utiliza dois algoritmos de verificação
de assinatura distintos, se usados algoritmos diferentes para certificados CA e
de folha. NTRU se destaca com um tamanho de código superior a 200 kB quando
usado para troca de chaves efêmeras, mas não quando utilizado apenas para
autenticação. O certificado CA com chave pública Rainbow ocupa entre 33% e 53%
do espaço de armazenamento devido ao seu grande tamanho, mas Rainbow ainda é
viável em sistemas embarcados devido ao tamanho pequeno das assinaturas e tempos
rápidos de verificação.

\begin{figure}[H]
	{\includegraphics[scale=0.4,page=5]{\pdf}}
	\caption{Comparação KEMTLS e PQTLS nível I NIST}
	\note{Tamanhos de código e certificado CA (e como porcentagem do tamanho total
		da ROM), e uso máximo de memória nos experimentos. Os conjuntos de parâmetros
		usados são do nível I do NIST.}
	\label{f:table3}\end{figure}

\paragraph{}
Além disso, os esquemas baseados em redes (lattice-based) têm bom desempenho em
termos de consumo de memória, que é principalmente impulsionado pelo uso da
pilha dos algoritmos de assinatura PQC. A exceção é Rainbow, cuja chave pública
grande precisa ser mantida na memória durante a verificação de assinatura,
exigindo grande alocação de heap. Seria possível otimizar isso armazenando a
chave pública já em um formato utilizável na flash, mas essa otimização não foi
incluída para permitir resultados comparáveis de código reutilizável.

% Results 4.2
\subsection{Termos de Handshake}

\paragraph{}
Os tempos de handshake são fundamentais em ambientes embarcados, assim como o
consumo de armazenamento e memória. A Tabela 4 apresenta os tempos de handshake
medidos em milhões de ciclos para diferentes tecnologias de transmissão, e a
Figura 2 ilustra esses tempos e o tráfego para cenários de banda larga e NB-IoT.
Em uma implantação real, o dispositivo poderia entrar em modo de baixo consumo
durante transmissões lentas, dependendo das especificações do sistema.

\paragraph{}
Nos experimentos, o CPU operou a 72 MHz para garantir resultados consistentes. A
Tabela 4 também destaca a porcentagem de ciclos dedicados às primitivas de
criptografia pós-quântica (PQC), enquanto os ciclos restantes foram utilizados
pela máquina de estado TLS, operações de memória ou espera por I/O.

\paragraph{}
Em configurações de banda larga e LTE-M, as operações criptográficas têm um
impacto significativo. No entanto, na NB-IoT, essas operações representam uma
pequena porcentagem do tempo total (0,8% - 1,7%). Nessas configurações de baixa
largura de banda e alta latência, o tamanho das transmissões de certificados e
chaves públicas é o principal fator que determina o tempo de execução. Além
disso, o carregamento de grandes chaves públicas da memória afeta a eficiência
do algoritmo Rainbow, que, de outra forma, seria rápido.

\begin{figure}[H]
	{\includegraphics[scale=0.4,page=6]{\pdf}}
	\caption{Tempo de handshake e execução}
	\note{Tráfego de handshake e tempo de execução para vários cenários. Os
		conjuntos de parâmetros usados são do nível I do NIST.}
	\label{f:table4}\end{figure}

\paragraph{}
Os KEMs superam o Dilithium na autenticação devido ao tempo de verificação mais
rápido e ao menor tamanho de chave e assinatura, o que reduz o tempo de
transmissão. O Rainbow, quando usado como certificado de autoridade
certificadora (CA), oferece os tempos de handshake mais rápidos, especialmente
em KEMTLS. O Falcon se destaca no servidor, mas tem operações de assinatura mais
lentas que os KEMs. Em cenários de banda larga e LTE-M, o desempenho do PQTLS
com Falcon e SABER é comparável ao do KEMTLS com Rainbow e SABER.

\begin{figure}[H]
	{\includegraphics[scale=0.4,page=7]{\pdf}}
	\caption{Tempo de handshake e tráfego para instâncias}
	\note{Tempos de handshake e tráfego para instâncias de KEMTLS e PQTLS. As
		letras representam os algoritmos (D)ilithium, (F)alcon, (K)yber, (N)TRU,
		(R)ainbow e (S)ABER nos papéis de troca de chaves efêmeras, autenticação de
		handshake e CA, nessa ordem.}
	\label{f:figure2}\end{figure}

% Discussion 5
\section{Discussão}

\paragraph{}
Os resultados mostram que o KEMTLS com autenticação apenas do servidor consome
menos memória que o PQTLS e possui tamanhos de código semelhantes. O PQTLS com
Falcon se destaca em termos de eficiência de verificação, superando outras
instâncias, exceto em cenários com SABER ou NTRU combinados com Rainbow, onde o
KEMTLS é mais eficiente devido ao menor uso de memória. Falcon também se sai
melhor que Dilithium no lado do cliente em qualquer cenário, especialmente em
ambientes de largura de banda muito baixa, como NB-IoT. No entanto, a assinatura
do Falcon pode não ser adequada para autenticação pós-quântica sem suporte de
hardware, onde uma combinação de Dilithium e Falcon pode ser mais eficaz

% Conclusion and Future Work 6
\section{Conclusão e Trabalho Futuro}

\paragraph{}
Neste artigo, comparou-se o desempenho de KEMTLS e TLS 1.3 usando finalistas de
criptografia pós-quântica (PQC) do NIST em um ambiente embarcado com um cliente
Cortex-M4 e um servidor de classe desktop. O KEMTLS consome menos memória que o
TLS 1.3 devido ao menor uso de memória dos KEMs, embora o tamanho do código seja
similar. As primitivas PQC exigem mais de 50% do tempo de handshake em ambos os
protocolos. No ambiente NB-IoT, as assinaturas pequenas do Rainbow são
vantajosas, enquanto o Falcon oferece bom desempenho devido à sua verificação
eficiente, apesar de ser caro para assinaturas. Futuras pesquisas devem explorar
a autenticação do cliente e servidores embarcados, e a redução de largura de
banda com chaves pré-distribuídas.

% \bibliography{bibliography}

\end{document}
